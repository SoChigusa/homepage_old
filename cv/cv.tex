\documentclass[12pt]{article}
\usepackage{epsf}
\usepackage{amsmath,amssymb}
\usepackage{bm}

\usepackage[dvipdfmx]{graphicx}

\usepackage{comment}
\usepackage{multirow}
\usepackage{braket}

\usepackage[dvipdfmx]{hyperref}
\usepackage{tgtermes}

\setlength{\textwidth}{16.5cm}
\setlength{\textheight}{21.5cm}
\setlength{\oddsidemargin}{0cm}
\setlength{\evensidemargin}{0cm}
\setlength{\topmargin}{0cm}
\setlength{\footskip}{0cm}

\renewcommand{\arraystretch}{1.2}
\renewcommand{\baselinestretch}{1.1}

\renewcommand{\topfraction}{1.0}
\renewcommand{\bottomfraction}{1.0}

\renewcommand{\refname}{Publications}

\allowdisplaybreaks[1]

\title{\vspace{-2cm}\textbf{Curriculum Vitae}}
\author{So Chigusa}

\begin{document}
\large
\maketitle

\newcommand{\lsim}{\stackrel{<}{_\sim}}
\newcommand{\gsim}{\stackrel{>}{_\sim}}

\newcommand{\rem}[1]{{$\spadesuit$\bf #1$\spadesuit$}}

% \renewcommand{\theequation}{\thesection.\arabic{equation}}

\renewcommand{\thefootnote}{\arabic{footnote})}
\setcounter{footnote}{0}

\vspace{-5mm}
\subsection*{Personal Data}

\vspace{-3mm}

\begin{table}[h]
 \begin{tabular}{ll}
  First Name: & So % (颯)
      \\
  Last Name: & Chigusa % (千草)
      \\
  Date of Birth: & May 22, 1992 \\
  Place of Birth: & Kobe, Japan \\
  Nationality: & Japanese \\
  % Age below
  Age: & 28 \\
  % Age above
  Sex: & Male \\
 \end{tabular}
\end{table}

\vspace{-5mm}
\begin{table}[h]
 \begin{tabular}{ll}
  Affiliation: & High Energy Accelerator Research Organization (KEK) \\
  Postcode: & 305-0801 \\
  Address: & 1-1, Oho, Tsukuba, Ibaraki \\
  Phone: & +81-29-864-1171 \\
  E-mail: &
      \href{mailto:schigusa@post.kek.jp}{schigusa@post.kek.jp}
      \\
  Homepage: & \url{https://sochigusa.bitbucket.io} \\
 \end{tabular}
\end{table}
\vspace{-5mm}

\subsection*{Education}
\vspace{-3mm}
\begin{table}[h]
 \begin{tabular}{lll}
  \hline \hline
  Date & Degree & Institution \\ \hline
  Mar. 23, 2020 & Doctor of Philosophy (Physics) & University of Tokyo \\
  Mar. 23, 2017 & Master of Science (Physics) & University of Tokyo \\
  Mar. 25, 2015 & Bachelor of Science (Physics) & University of Tokyo \\
  \hline \hline
 \end{tabular}
\end{table}

\newpage
\subsection*{Professional experience}
\vspace{-3mm}
\begin{table}[h]
 \begin{tabular}{ll}
  \begin{tabular}{l}
   Apr. 2020 -- :
  \end{tabular} &
  \begin{tabular}{l}
   Postdoc, High Energy Accelerator Research Organization (KEK)
  \end{tabular}\\
  \begin{tabular}{l}
   Apr. 2015 -- Mar. 2020 :\\
  \quad
  \end{tabular} &
  \begin{tabular}{l}
   Ph.D. Student, Department of Physics, University of Tokyo\\
   (Dr. Takeo Moroi)
  \end{tabular}
 \end{tabular}
\end{table}
\vspace{-5mm}

\subsection*{Teaching experience}
\vspace{-3mm}
\begin{table}[h]
 \begin{tabular}{ll}
  \begin{tabular}{l}
   Apr. 2015 -- Sep. 2015 :\\
   \quad
  \end{tabular} &
  \begin{tabular}{l}
   Teaching Assistant for Undergraduate Class ``Quantum Mechanics II''\\
   at Department of Physics, University of Tokyo
  \end{tabular}
 \end{tabular}
\end{table}
\vspace{-5mm}

\subsection*{Grants}
\vspace{-3mm}
\begin{table}[h]
 \begin{tabular}{ll}
  \begin{tabular}{l}
   Apr. 2020 -- :
  \end{tabular} &
  \begin{tabular}{l}
   JSPS, Research Fellowships for Young Scientists (PD)
  \end{tabular}\\
  \begin{tabular}{l}
   Apr. 2017 -- Mar. 2020 :\\
   \quad
  \end{tabular} &
  \begin{tabular}{l}
   JSPS, Research Fellowships for Young Scientists (DC1)\\
   Amount: 2800000 JPY
  \end{tabular}\\
  \begin{tabular}{l}
   Oct. 2015 -- Mar. 2020 :
  \end{tabular} &
  \begin{tabular}{l}
   MEXT, Program for Leading Graduate Schools
  \end{tabular}
 \end{tabular}
\end{table}
\vspace{-5mm}

\subsection*{Honors and Awards}
\begin{enumerate}
 % Awards below
 \item Best presentation award for young scientists @ Unraveling the History of the Universe 2020
 \item Best Poster Award @ HPNP 2019
 % Awards above
\end{enumerate}

\bibliographystyle{JHEP}
\bibliography{cv}
\nocite{*}

\subsection*{Invited Seminar Presentations}
\begin{enumerate}
 % Seminars below
 \item ``固体中の「アクシオン」を用いた軽いボソン暗黒物質の直接探索'', 2021/3/1, Toyama, Kanazawa University
 \item ``Detecting Light Boson Dark Matter through Conversion into Magnon (Online)'', 2020/6/22, Nagoya University
 \item ``Detecting Light Boson Dark Matter through Conversion into Magnon (Online)'', 2020/6/12, UC Berkeley
 \item ``Detecting Light Boson Dark Matter through Conversion into Magnon (Online)'', 2020/6/2, Kyushu University
 \item ``Detecting Light Boson Dark Matter through Conversion into Magnon (Online)'', 2020/5/20, IBS
 \item ``Detecting Light Boson Dark Matter through Conversion into Magnon (Online)'', 2020/5/14, TDLI and INPAC
 \item ``Flowing to the Bounce'', 2019/10/24, Tohoku University
 \item ``Indirect Studies of Electroweakly Interacting Particles at 100 TeV Hadron Colliders'', 2019/7/23, Osaka University
 \item ``Indirect Studies of Electroweakly Interacting Particles at 100 TeV Hadron Colliders'', 2019/5/16, University of Florida
 \item ``Indirect Studies of Electroweakly Interacting Particles at 100 TeV Hadron Colliders'', 2019/5/10, Florida State University
 \item ``Indirect Studies of Electroweakly Interacting Particles at 100 TeV Hadron Colliders'', 2019/4/9, KEK
 \item ``Solutions to Domain Wall Problem in Models with Discrete Flavor Symmetry'', 2019/1/11, Hokkaido University
 \item ``Probing Electroweakly Interacting Massive Particles with Drell-Yan Process at 100 TeV Hadron Colliders'', 2018/10/16, Nagoya University
 % Seminars above
\end{enumerate}

\subsection*{Presentations at International Conferences}
\subsubsection*{(Oral)}
\begin{enumerate}
 % International Oral below
 \item ``Anomaly Mediation at Future Hadron Colliders'', 2020/8/4, KEK-PH 2020, Tsukuba
 \item ``Flowing to the Bounce'', 2020/1/14, Berkeley Week, IPMU
 \item ``Indirect Studies of Electroweakly Interacting Particles at 100 TeV Hadron Colliders'', 2019/8/20, SI2019, Gangneung, Korea
 \item ``Flowing to the Bounce'', 2019/8/9, NHWG26, Osaka
 \item ``Indirect Studies of Electroweakly Interacting Particles at 100 TeV Hadron Colliders'', 2019/5/22, SUSY 2019, Texas
 \item ``Indirect Studies of Electroweakly Interacting Particles at 100 TeV Hadron Colliders'', 2019/5/6, Pheno 2019, Pittsburgh
 \item ``Flavon Stabilization in Models with Discrete Flavor Symmetry'', 2018/12/6, KEK-PH 2018 winter, Tsukuba
 \item ``Decay Rate of the Electroweak Vacuum in the Standard Model and Beyond'', 2018/5/24, Planck 2018, Bonn
 \item ``Bottom-Tau Unification in Supersymmetric Models'', 2017/2/6, New Physics Forum, IPMU
 \item ``Bottom-Tau unification in Supersymmetric Model with Anomaly-Mediation'', 2016/7/05, SUSY 2016, Melbourne
 % International Oral above
\end{enumerate}
\subsubsection*{(Poster)}
\begin{enumerate}
 % International Poster below
 \item ``Probing Electroweakly Interacting Massive Particles with Precision Measurements at 100 TeV Hadron Colliders (poster)'', 2019/2/21, HPNP2019, Osaka
 % International Poster above
\end{enumerate}

\subsection*{Presentations at Domestic Conferences}
\subsubsection*{(Oral)}
\begin{enumerate}
 % Domestic Oral below
 \item ``スピン励起を用いた軽いボソン暗黒物質の直接探索'', 2021/3/31, KEK「素核宇・物性」連携研究会, ホーム
 \item ``XENON1T 実験の結果を説明する模型への制限'', 2020/9/8, ダークマターの懇談会2020 online, Online
 \item ``特徴的なシグナルを用いた暗黒物質模型の探索(招待講演)'', 2020/8/11, 新テラスケール研究会, Online
 \item ``マグノンを用いた軽いボソン暗黒物質の直接探索'', 2020/6/2, Unraveling the History of the Universe 2020, Online
 \item ``Flavon Stabilization without Domain Wall Problem in Discrete Flavor Symmetry Models (in Japanese)'', 2019/6/11, Neutrino Oscillation and Flavor Physics, Nagoya
 \item ``Zero Mode Problem in the Calculation of Decay Rate of the SM Electroweak vacuum'', 2018/9/15, JPS 2018, Shinshu
 \item ``Bottom-Tau unification in Supersymmetric Model with Anomaly-Mediation'', 2016/9/21, JPS 2016, Miyazaki
 % Domestic Oral above
\end{enumerate}
\subsubsection*{(Poster)}
\begin{enumerate}
 % Domestic Poster below
 \item ``Indirect Search of WIMP Dark Matter at Future 100 TeV Collider (Poster)'', 2018/8/9, PPP 2018, Kyoto
 \item ``Bottom Tau Unification in Supersymmetric Models (Poster)'', 2017/8/3, PPP 2017, Kyoto
 % Domestic Poster above
\end{enumerate}

\subsection*{Poster Presentations at International Summer Schools}
\begin{enumerate}
  % Summer School below
 \item ``Decay Rate of the Electroweak Vacuum in the Standard Model and Beyond'', 2018/7/12, Cargese Summer School 2018, Kyoto
 \item ``Bottom Tau Unification in Supersymmetric Models (Poster)'', 2017/7/4, Les Houches Summer School 2017, Kyoto
  % Summer School above
\end{enumerate}

\end{document}
